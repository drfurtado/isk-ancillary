% Options for packages loaded elsewhere
\PassOptionsToPackage{unicode}{hyperref}
\PassOptionsToPackage{hyphens}{url}
%
\documentclass[
]{article}
\usepackage{amsmath,amssymb}
\usepackage{lmodern}
\usepackage{iftex}
\ifPDFTeX
  \usepackage[T1]{fontenc}
  \usepackage[utf8]{inputenc}
  \usepackage{textcomp} % provide euro and other symbols
\else % if luatex or xetex
  \usepackage{unicode-math}
  \defaultfontfeatures{Scale=MatchLowercase}
  \defaultfontfeatures[\rmfamily]{Ligatures=TeX,Scale=1}
\fi
% Use upquote if available, for straight quotes in verbatim environments
\IfFileExists{upquote.sty}{\usepackage{upquote}}{}
\IfFileExists{microtype.sty}{% use microtype if available
  \usepackage[]{microtype}
  \UseMicrotypeSet[protrusion]{basicmath} % disable protrusion for tt fonts
}{}
\makeatletter
\@ifundefined{KOMAClassName}{% if non-KOMA class
  \IfFileExists{parskip.sty}{%
    \usepackage{parskip}
  }{% else
    \setlength{\parindent}{0pt}
    \setlength{\parskip}{6pt plus 2pt minus 1pt}}
}{% if KOMA class
  \KOMAoptions{parskip=half}}
\makeatother
\usepackage{xcolor}
\usepackage[margin=1in]{geometry}
\usepackage{longtable,booktabs,array}
\usepackage{calc} % for calculating minipage widths
% Correct order of tables after \paragraph or \subparagraph
\usepackage{etoolbox}
\makeatletter
\patchcmd\longtable{\par}{\if@noskipsec\mbox{}\fi\par}{}{}
\makeatother
% Allow footnotes in longtable head/foot
\IfFileExists{footnotehyper.sty}{\usepackage{footnotehyper}}{\usepackage{footnote}}
\makesavenoteenv{longtable}
\usepackage{graphicx}
\makeatletter
\def\maxwidth{\ifdim\Gin@nat@width>\linewidth\linewidth\else\Gin@nat@width\fi}
\def\maxheight{\ifdim\Gin@nat@height>\textheight\textheight\else\Gin@nat@height\fi}
\makeatother
% Scale images if necessary, so that they will not overflow the page
% margins by default, and it is still possible to overwrite the defaults
% using explicit options in \includegraphics[width, height, ...]{}
\setkeys{Gin}{width=\maxwidth,height=\maxheight,keepaspectratio}
% Set default figure placement to htbp
\makeatletter
\def\fps@figure{htbp}
\makeatother
\setlength{\emergencystretch}{3em} % prevent overfull lines
\providecommand{\tightlist}{%
  \setlength{\itemsep}{0pt}\setlength{\parskip}{0pt}}
\setcounter{secnumdepth}{5}
\newlength{\cslhangindent}
\setlength{\cslhangindent}{1.5em}
\newlength{\csllabelwidth}
\setlength{\csllabelwidth}{3em}
\newlength{\cslentryspacingunit} % times entry-spacing
\setlength{\cslentryspacingunit}{\parskip}
\newenvironment{CSLReferences}[2] % #1 hanging-ident, #2 entry spacing
 {% don't indent paragraphs
  \setlength{\parindent}{0pt}
  % turn on hanging indent if param 1 is 1
  \ifodd #1
  \let\oldpar\par
  \def\par{\hangindent=\cslhangindent\oldpar}
  \fi
  % set entry spacing
  \setlength{\parskip}{#2\cslentryspacingunit}
 }%
 {}
\usepackage{calc}
\newcommand{\CSLBlock}[1]{#1\hfill\break}
\newcommand{\CSLLeftMargin}[1]{\parbox[t]{\csllabelwidth}{#1}}
\newcommand{\CSLRightInline}[1]{\parbox[t]{\linewidth - \csllabelwidth}{#1}\break}
\newcommand{\CSLIndent}[1]{\hspace{\cslhangindent}#1}
\ifLuaTeX
  \usepackage{selnolig}  % disable illegal ligatures
\fi
\IfFileExists{bookmark.sty}{\usepackage{bookmark}}{\usepackage{hyperref}}
\IfFileExists{xurl.sty}{\usepackage{xurl}}{} % add URL line breaks if available
\urlstyle{same} % disable monospaced font for URLs
\hypersetup{
  pdftitle={Statistical Methods},
  pdfauthor={Furtado Jr, Ovande},
  hidelinks,
  pdfcreator={LaTeX via pandoc}}

\title{Statistical Methods}
\author{\href{http://drfurtado.us}{Furtado Jr, Ovande}}
\date{Last updated on 2022-04-11}

\begin{document}
\maketitle

{
\setcounter{tocdepth}{2}
\tableofcontents
}
\hypertarget{tests---one-sample-z-test}{%
\section{Tests - One-Sample z-test}\label{tests---one-sample-z-test}}

\hypertarget{readme}{%
\subsection{Readme}\label{readme}}

Other formats: \href{2means.pdf}{PDF} \textbar{} \href{2means.html}{HTML}

Hint: use Control+F (Windows) or Command+F (Mac) to search this page

\begin{center}\rule{0.5\linewidth}{0.5pt}\end{center}

\hypertarget{learning-objectives}{%
\subsection{Learning objectives}\label{learning-objectives}}

\begin{enumerate}
\def\labelenumi{\arabic{enumi}.}
\tightlist
\item
  sds
\end{enumerate}

\hypertarget{when-to-use-it}{%
\subsection{When to use it?}\label{when-to-use-it}}

\begin{itemize}
\tightlist
\item
  Comparing a sample mean to population mean
\item
  Standard deviation of the population is known
\item
  Have a single quantitative variable of interval/ratio level
\end{itemize}

\hypertarget{stating-the-hypotheses}{%
\subsection{Stating the Hypotheses}\label{stating-the-hypotheses}}

\textbf{Null hypothesis}

\(H_0:\mu = \mu_0\)

where, \(\mu\) is the population mean and \(\mu_0\) refer to some specified value.

\textbf{Alternative hypothesis}

\(H_1\) two sided: \(\mu \ne \mu_0\)

\(H_1\) right sided: \(\mu > \mu_0\)

\(H_1\) left sided: \(\mu < \mu_0\)

\hypertarget{assumptions}{%
\subsection{Assumptions}\label{assumptions}}

\begin{itemize}
\tightlist
\item
  The data are continuous (not discrete);
\item
  The data follow the normal probability distribution;
\item
  Population standard deviation \(\sigma\) is known;
\item
  Sample is a simple random sample from the population\footnote{Each individual in the population has an equal probability of being selected in the sample.}.
\end{itemize}

\hypertarget{test-statistic}{%
\subsection{Test statistic}\label{test-statistic}}

\[
z = \dfrac{\bar{X} - \mu_0}{\sigma / \sqrt{N}}
\] where, \(\bar{X}\) is the sample mean, \(\mu_0\) is the population mean according to the null hypothesis, \(\sigma\) is the population standard deviation, and N is the sample size.

\hypertarget{sampling-distribution}{%
\subsection{Sampling distribution}\label{sampling-distribution}}

When running the z-test, you will need to use the \href{https://statkat.com/sampling-distribution/one-sample-z-test/z.php}{sampling distribution of z}, which is the standard normal distribution.

\begin{center}\rule{0.5\linewidth}{0.5pt}\end{center}

\emph{Suppose that the assumptions of the one sample z test hold, and that the null hypothesis that} \(\mu\) \emph{=} \(\mu_0\) \emph{is true. Then the sampling distribution of z is normal with mean 0 and standard deviation 1 (standard normal). That is, most of the time we would find z values close to 0, and only sometimes we would find z values further away from 0. If we find a z value in our actual sample that is far away from 0, this is a rare event if the null hypothesis were true, and is therefore considered evidence against the null hypothesis (z value in rejection region, small p value)}\textsuperscript{{[}1{]}}.

\begin{center}\rule{0.5\linewidth}{0.5pt}\end{center}

\hypertarget{significance}{%
\subsection{Significance}\label{significance}}

To find out whether the test is significant, by comparing the observed test statistics (z value) against the critical value after considering the \textbf{alpha value}, the \textbf{type of test} (two-sided, right-sided, or left sided), and the \textbf{degrees of freedom}.

\begin{itemize}
\item
  compare the observed test statistic with the critical value of z (refer to Table \ref{tab:critical-values}.

  \begin{itemize}
  \tightlist
  \item
    if the observed z value is equal or greater than the critical value, reject the \(H_0\) ; or
  \end{itemize}
\item
  compare the observed p-value\footnote{Value calculated by the statistical package; i.e., jamovi, SPSS} against the alpha value (\(\alpha\)).

  \begin{itemize}
  \tightlist
  \item
    if the calculate p-value is less than the \(\alpha\), then reject the \(H_0\)
  \end{itemize}
\end{itemize}

\hypertarget{appendices}{%
\section*{Appendices}\label{appendices}}
\addcontentsline{toc}{section}{Appendices}

\hypertarget{rejection-regions}{%
\subsection*{Rejection regions}\label{rejection-regions}}
\addcontentsline{toc}{subsection}{Rejection regions}

\begin{figure}
\hypertarget{figure1}{%
\centering
\includegraphics{https://lsj.readthedocs.io/en/latest/_images/lsj_zTestOneTwoTailed.svg}
\caption{\label{fig:rejection-regions} Rejection regions for the two-sided z-test (left panel) and the one-sided z-test (right panel)}\label{figure1}
}
\end{figure}

\hypertarget{critical-values}{%
\subsection*{Critical values}\label{critical-values}}
\addcontentsline{toc}{subsection}{Critical values}

\begin{longtable}[]{@{}
  >{\raggedright\arraybackslash}p{(\columnwidth - 4\tabcolsep) * \real{0.4203}}
  >{\raggedright\arraybackslash}p{(\columnwidth - 4\tabcolsep) * \real{0.2899}}
  >{\raggedright\arraybackslash}p{(\columnwidth - 4\tabcolsep) * \real{0.2899}}@{}}
\caption{\label{tab:critical-values} Critical values based on desired \(\alpha\) level)}\tabularnewline
\toprule()
\begin{minipage}[b]{\linewidth}\raggedright
\textbf{desired} alpha \textbf{level}
\end{minipage} & \begin{minipage}[b]{\linewidth}\raggedright
\textbf{two-sided test}
\end{minipage} & \begin{minipage}[b]{\linewidth}\raggedright
\textbf{one-sided test}
\end{minipage} \\
\midrule()
\endfirsthead
\toprule()
\begin{minipage}[b]{\linewidth}\raggedright
\textbf{desired} alpha \textbf{level}
\end{minipage} & \begin{minipage}[b]{\linewidth}\raggedright
\textbf{two-sided test}
\end{minipage} & \begin{minipage}[b]{\linewidth}\raggedright
\textbf{one-sided test}
\end{minipage} \\
\midrule()
\endhead
0.1 & 1.644854 & 1.281552 \\
0.05 & 1.959964 & 1.644854 \\
0.01 & 2.575829 & 2.326348 \\
0.001 & 3.290527 & 3.090232 \\
\bottomrule()
\end{longtable}

\hypertarget{references}{%
\section*{References}\label{references}}
\addcontentsline{toc}{section}{References}

\hypertarget{refs}{}
\begin{CSLReferences}{1}{0}
\leavevmode\vadjust pre{\hypertarget{ref-rivkadevries}{}}%
1. Rivka deVries. (n.d.). \emph{Sampling distribution of the z statistic - one sample z test}. \url{https://statkat.com/sampling-distribution/one-sample-z-test/z.php}

\end{CSLReferences}

\end{document}
