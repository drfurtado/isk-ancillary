% Options for packages loaded elsewhere
\PassOptionsToPackage{unicode}{hyperref}
\PassOptionsToPackage{hyphens}{url}
%
\documentclass[
]{article}
\usepackage{amsmath,amssymb}
\usepackage{lmodern}
\usepackage{iftex}
\ifPDFTeX
  \usepackage[T1]{fontenc}
  \usepackage[utf8]{inputenc}
  \usepackage{textcomp} % provide euro and other symbols
\else % if luatex or xetex
  \usepackage{unicode-math}
  \defaultfontfeatures{Scale=MatchLowercase}
  \defaultfontfeatures[\rmfamily]{Ligatures=TeX,Scale=1}
\fi
% Use upquote if available, for straight quotes in verbatim environments
\IfFileExists{upquote.sty}{\usepackage{upquote}}{}
\IfFileExists{microtype.sty}{% use microtype if available
  \usepackage[]{microtype}
  \UseMicrotypeSet[protrusion]{basicmath} % disable protrusion for tt fonts
}{}
\makeatletter
\@ifundefined{KOMAClassName}{% if non-KOMA class
  \IfFileExists{parskip.sty}{%
    \usepackage{parskip}
  }{% else
    \setlength{\parindent}{0pt}
    \setlength{\parskip}{6pt plus 2pt minus 1pt}}
}{% if KOMA class
  \KOMAoptions{parskip=half}}
\makeatother
\usepackage{xcolor}
\usepackage[margin=1in]{geometry}
\usepackage{longtable,booktabs,array}
\usepackage{calc} % for calculating minipage widths
% Correct order of tables after \paragraph or \subparagraph
\usepackage{etoolbox}
\makeatletter
\patchcmd\longtable{\par}{\if@noskipsec\mbox{}\fi\par}{}{}
\makeatother
% Allow footnotes in longtable head/foot
\IfFileExists{footnotehyper.sty}{\usepackage{footnotehyper}}{\usepackage{footnote}}
\makesavenoteenv{longtable}
\usepackage{graphicx}
\makeatletter
\def\maxwidth{\ifdim\Gin@nat@width>\linewidth\linewidth\else\Gin@nat@width\fi}
\def\maxheight{\ifdim\Gin@nat@height>\textheight\textheight\else\Gin@nat@height\fi}
\makeatother
% Scale images if necessary, so that they will not overflow the page
% margins by default, and it is still possible to overwrite the defaults
% using explicit options in \includegraphics[width, height, ...]{}
\setkeys{Gin}{width=\maxwidth,height=\maxheight,keepaspectratio}
% Set default figure placement to htbp
\makeatletter
\def\fps@figure{htbp}
\makeatother
\setlength{\emergencystretch}{3em} % prevent overfull lines
\providecommand{\tightlist}{%
  \setlength{\itemsep}{0pt}\setlength{\parskip}{0pt}}
\setcounter{secnumdepth}{5}
\ifLuaTeX
  \usepackage{selnolig}  % disable illegal ligatures
\fi
\IfFileExists{bookmark.sty}{\usepackage{bookmark}}{\usepackage{hyperref}}
\IfFileExists{xurl.sty}{\usepackage{xurl}}{} % add URL line breaks if available
\urlstyle{same} % disable monospaced font for URLs
\hypersetup{
  pdftitle={KIN-610 - Stats Project},
  pdfauthor={Furtado, JR., O},
  hidelinks,
  pdfcreator={LaTeX via pandoc}}

\title{KIN-610 - Stats Project}
\author{\href{http://drfurtado.us}{Furtado, JR., O}}
\date{Last updated on 2022-05-05}

\begin{document}
\maketitle

{
\setcounter{tocdepth}{2}
\tableofcontents
}
\hypertarget{readme}{%
\section{Readme}\label{readme}}

Other formats: \href{project1.pdf}{PDF} \textbar{} \href{project1.html}{HTML}

Hint: use Control+F (Windows) or Command+F (Mac) to search this page

\begin{center}\rule{0.5\linewidth}{0.5pt}\end{center}

\hypertarget{learning-objectives}{%
\section{Learning objectives}\label{learning-objectives}}

\begin{enumerate}
\def\labelenumi{\arabic{enumi}.}
\tightlist
\item
  Identify variables from a research question statement
\item
  Formulate hypotheses from a research question statement
\item
  Identify the appropriate statistical procedure to test the formulated hypothesis
\item
  Conduct statistical analysis in \texttt{jamovi}
\item
  Interpret the results of the data analysis
\item
  Create tables and figures to illustrate the findings
\item
  Prepare a research report
\end{enumerate}

\hypertarget{data}{%
\section{Data}\label{data}}

\hypertarget{data-set}{%
\subsection{Data set}\label{data-set}}

\begin{enumerate}
\def\labelenumi{\arabic{enumi}.}
\tightlist
\item
  Download the data set: \url{https://osf.io/8ehtw/}
\item
  Download the data codebook: \url{https://osf.io/b87a3/}
\end{enumerate}

\hypertarget{data-info}{%
\subsection{Data Information}\label{data-info}}

The date for this project come from the \href{https://en.wikipedia.org/wiki/NFL_Scouting_Combine}{NFL Scouting Combine}. The NFL Combine is held prior to the draft every year, testing players in the 40 yard dash, vertical jump, bench press, broad jump, shuttle, and three cone drill.

The description of each drill can be \href{https://nflcombineresults.com/nfl-combine-drills-101-what-each-drill-measures/}{found here} and \href{https://www.espn.com/nfl/draft2018/story/_/id/22587931/guide-nfl-draft-combine-drills-todd-mcshay-numbers-know-40-yard-dash-short-shuttle-bench-press}{here}. Please, \href{https://protips.dickssportinggoods.com/sports-and-activities/football/football-101-football-positions-and-their-roles}{click here} to learn about football positions.

\begin{quote}
Note: This is NOT a true random sample since not everyone from the study population had the same chance to be included in the sample. This is important when deciding the appropriate statistical procedure (model) needed to test the hypothesis(ses).
\end{quote}

\hypertarget{data-modifications}{%
\subsection{Data modifications}\label{data-modifications}}

For the final project, I have modified the data set in the following ways:

\begin{enumerate}
\def\labelenumi{\arabic{enumi}.}
\item
  Extra variables were created for each dependent variable and random numbers were generated for these variables. Thus, the data for the ``pos-test'' variables are fabricated.
\item
  Scores were also fabricated using the test's population mean and standard deviation for the Wonderlic\footnote{Wonderlic test - Wikipedia.'' \url{https://en.wikipedia.org/wiki/Wonderlic_test}. Accessed 21 Oct.~2020.} variable. Due to privacy, the data for this variable are not publicly available.
\end{enumerate}

\hypertarget{statistical-package}{%
\subsection{Statistical package}\label{statistical-package}}

Although students are free to use the software of their choice, I recommend \href{www.jamovi.org}{\texttt{jamovi}}, which is covered extensively in my course.

\hypertarget{steps}{%
\section{Required steps}\label{steps}}

In a nutshell, you will be assigned a \protect\hyperlink{appendix-a}{research question} and be given a data set. Then, after running the appropriate statistical model, you can proceed to write your research report.

\hypertarget{prelim-info}{%
\subsection{Preliminary information}\label{prelim-info}}

\emph{Hint: copy the \protect\hyperlink{templates}{template} associated with this assignment and paste it into a Google Doc document.}

Before running the statistical test to answer your research question, you will be asked to complete these steps:

\textbf{Part 1 - steps 1-3}

\begin{enumerate}
\def\labelenumi{\arabic{enumi}.}
\tightlist
\item
  Identify the DV and IV
\item
  State the Null (\(H_0\)) and the alternate hypotheses (\(H_1\))
\item
  Choose the statistical model to test your Null hypothesis (\(H_0\))
\end{enumerate}

\textbf{Part 2 - step 4}

\begin{enumerate}
\def\labelenumi{\arabic{enumi}.}
\setcounter{enumi}{3}
\tightlist
\item
  Search related articles + 2-page summary
\end{enumerate}

\hypertarget{part-1}{%
\subsubsection{Part 1}\label{part-1}}

\hypertarget{dependent-dv-and-independent-variable-iv}{%
\paragraph{Dependent (DV) and independent variable (IV)}\label{dependent-dv-and-independent-variable-iv}}

The assignment of research question will be done on the first day of class and a record of the assignment will be available on Canvas (Modules \textgreater{} Project Information).

Once you are assigned a number, refer to \protect\hyperlink{appendix-b}{Appendix B} to identify the research question associated with your assigned \#.

Once the research question is identified, proceed to identify the independent and dependent variables.

\begin{enumerate}
\def\labelenumi{\alph{enumi}.}
\item
  For the DV, identify whether it is \emph{continuous} or \emph{discrete} (nominal or ordinal).
\item
  For the IV, identify the levels associated with it (i.,e. sex - two levels; males and females)
\end{enumerate}

\begin{center}\rule{0.5\linewidth}{0.5pt}\end{center}

Whenever applicable, run the normality plots with tests to verify whether the distribution of scores for the dependent variable is approximating or deviating from normality. Use this information to decide whether you will need to use a parametric or non-parametric test (i., e. independent-samples t-test vs Mann-Whitney test). If you do, then provide the information under Data Analysis.

\begin{center}\rule{0.5\linewidth}{0.5pt}\end{center}

\hypertarget{hypothesisses-for-each-of-the-dependent-variables}{%
\paragraph{Hypothesis(ses) for each of the dependent variable(s)}\label{hypothesisses-for-each-of-the-dependent-variables}}

A hypothesis for each of the DVs must be formulated. First, state your prediction; i.e., \texttt{10-year-old\ girls\ will\ overperform\ 10-year-old\ boys\ on\ the\ skill\ of\ skipping}.

Next, you must decide whether you will be testing your hypothesis using the one-tailed or the two-tailed test. Once this is decided, you are ready to state the \texttt{Ho} (Null) and the \texttt{H1} (alternate Hypothesis). Below is an example of two-tailed test (non-directional hypothesis):

\begin{itemize}
\tightlist
\item
  \(H_0\): 10-year-old girls and boys perform similarly on the skill of skipping
\item
  \(H_a\): 10-year-old girls and boys will perform differently on the skill of skipping
\end{itemize}

Notice above that \texttt{Ha} was stated as non-directional (two-tailed) because, even though we predicted that girls will do better than boys, we are uncertain of this prediction. Thus, the recommendation is to choose the two-tailed test.

Recall that you should only use a directional hypothesis (one-tailed) if you have strong evidence of the direction of the effect (refer to our textbook about one- and two-tailed tests).

\hypertarget{choose-the-appropriate-statistical-model-to-test-the-hypothesisses}{%
\paragraph{Choose the appropriate statistical model to test the hypothesis(ses)}\label{choose-the-appropriate-statistical-model-to-test-the-hypothesisses}}

State the statistical procedure (i.e., ANOVA, Spearman rho) you selected to test the hypothesis(ses) and explain the rationale for choosing the procedure(s).

Recall that the selection of the procedure will depend on several factors, including but not limited to, \texttt{1)} the nature of the dependent variable (continuous, discrete), \texttt{2)} the level (nominal, ordinal, scale), \texttt{3)} the number of DVs/IVs. In addition, because the sample you are using is not a \texttt{true\ random\ sample}, normality cannot be assumed by default. You must run the normality plots with tests for each dependent variable and report it appropriately.

\begin{center}\rule{0.5\linewidth}{0.5pt}\end{center}

Hint: You may use the \href{https://statkat.com/index.php}{StatsKat} website to help you choosing the test.

\begin{center}\rule{0.5\linewidth}{0.5pt}\end{center}

Note that you must chose a statistical model for each research hypothesis ( \(H_a\) ) formulated.

\begin{center}\rule{0.5\linewidth}{0.5pt}\end{center}

\hypertarget{part-2}{%
\subsubsection{Part 2}\label{part-2}}

\hypertarget{articles}{%
\paragraph{Search for related articles}\label{articles}}

You will be asked to search and find at least \emph{3 articles} related to your research question. These articles are to be used when writing the \texttt{Introduction} section of your report. Ensure to include these three sources under \texttt{References}.

Once you find the articles, read, summarize and \href{https://bit.ly/2YIGK2t}{create annotated bibliography entries for each}. Note that you will be required to submit the links to the 3 articles and the annotated bibliography entries for each of the 3 articles before the \protect\hyperlink{deadlines}{deadline}.

Then, proceed to write a 2-page summary that will be part of the \texttt{Introduction} section of the final project. Note that you will need to turn in the 2-page summary by the \protect\hyperlink{deadlines}{deadline}.

\begin{center}\rule{0.5\linewidth}{0.5pt}\end{center}

\emph{Hint: Use the guidelines from Dr.~McLeod (\url{https://bit.ly/3C46nbw}) when writing the \texttt{Introduction}.}

\begin{center}\rule{0.5\linewidth}{0.5pt}\end{center}

\hypertarget{presenting-results}{%
\subsection{Presenting results}\label{presenting-results}}

You will be given 15 minutes to present the results of your project - 10 minutes (presenting) and 5 minutes (Q\&A).

You will be required to use PowerPoint, Google Slides, or something similar. I have created a \href{pres-template.html}{template} you can use to get started. You can also import it to Keynote, Google Slides if needed. Please, do a search online to learn how this can be accomplished.

\textbf{Presentation draft}

You will be required to turn-in a draft of the slides the week before your presentation. Use the template above to create the presentation draft. \textbf{On each of the slides of the template, indicate what needs to be added}. For instance,

\begin{verbatim}
- SLide 1: The title and my name will be added
- Slide 2: ....
- ...
- Slide 6: Statistical Analysis
  - The nature of the test
  - The variables of interest
  - etc...
\end{verbatim}

\hypertarget{written-report}{%
\subsection{Written report}\label{written-report}}

Once the preliminary information is gathered, and after running the appropriate statistical model to test your hypothesis(ses), you can proceed to write your research report. Below are the required sections of the written report.

A template was created following APA Style 7th ed.~Ensure to use the template when completing the written report.

The template can be found here --\textgreater{} \url{https://bit.ly/kin610-project-template}

\begin{verbatim}
1. Title
2. Abstract
 - help found here --> https://bit.ly/3Npz1e7
3. Introduction
 - help found here --> https://bit.ly/3C46nbw
4. Method 
 - help found here --> https://bit.ly/3iAR2rQ
   a. Participants
   b. Design
    - more help found here --> https://bit.ly/37TZdgE
   c. Statistical Analysis
    - help found here --> https://bit.ly/31skJ5I 
   c. Materials
    - help found here --> https://bit.ly/3iAR2rQ
   d. Procedures
   - help found here --> https://bit.ly/3iAR2rQ
5. Results
 - help found here --> https://bit.ly/3Lg9qT5
6. References
 - help found here --> https://bit.ly/3tHqIm7
7. Appendix
   7.1 Preliminary info
   - Add here the preliminary info turned in previously
\end{verbatim}

\begin{center}\rule{0.5\linewidth}{0.5pt}\end{center}

\emph{Hint 1: Download and read \href{https://bit.ly/31skJ5I}{this article} that covers the best practices of Methods/Results/Conclusion write-ups.}

\emph{Hint 2: Follow the example proposed by \href{https://bit.ly/3fptZzD}{Dr.~McLeod}.}

Hint 3: Ensure to phrase your results following the \href{https://bit.ly/2HirVLv}{APA Style}

\begin{center}\rule{0.5\linewidth}{0.5pt}\end{center}

\hypertarget{templates}{%
\section{Templates}\label{templates}}

Below you will find the templates you are to use when submitting the required assignments related to the final project.

\begin{enumerate}
\def\labelenumi{\arabic{enumi}.}
\tightlist
\item
  Copy the content of each template
\item
  Open \href{http://docs.google.com}{Google Docs} and paste the content into the body of the blank document - Do NOT change the font type or size
\item
  Proceed to complete the assignment, then click on ``File\textgreater Download\textgreater PDF Document''
\item
  Finally, submit the PDF file via Canvas
\end{enumerate}

\hypertarget{preliminary-info-part-1}{%
\subsection{Preliminary info (Part 1)}\label{preliminary-info-part-1}}

\textbf{Variables, Hypotheses, and Model}

\begin{verbatim}
Cal State Northridge - Kinesiology
KIN-610 - Dr. Furtado
Points: ___/30

YOUR RESEAERCH QUESTION
Type here...


DEPENDENT AND INDEPENDENT VARIABLES (10 points)

IV: 
Levels of the IV:
Scale:

DV:
Scale:

HYPOTHESES (10 points)

State your prediction (i.e., boys will perform better than gilrs on the skill of kicking)
Type here...

Type of hypothesis (e.g., directional vs non-directional)

State the null hyphothsis (Ho)
Type here...

State the alternative (aka research hypothesis) hyphothsis (Ha)
Type here...

STATISTICAL MODEL (10 points)

State the statistical model (test) that you chose to test the hypothesis
Type here...

Reason 1 for choosing the test:
Reason 2 for choosing the test:
Reason 3 for choosing the test:
\end{verbatim}

\hypertarget{preliminary-info-part-2}{%
\subsection{Preliminary info (Part 2)}\label{preliminary-info-part-2}}

\textbf{Articles and annotated bibliography entries}

\begin{verbatim}
Cal State Northridge - Kinesiology
KIN-610 - Dr. Furtado
Points ___/30

YOUR RESEAERCH QUESTION
Type here...

Link to 1st article:
URL here....

Annotated bibliography entry:
Type text here...


Link to 2nd article:
URL here....

Annotated bibliography entry:
Type text here...


Link to 3rd article:
URL here....

Annotated bibliography entry:
Type text here...

Note: visit the following link to see examples of annotated bibliography
entries: https://bit.ly/2YIGK2t
\end{verbatim}

\hypertarget{two-page-summary}{%
\subsection{Two-page summary}\label{two-page-summary}}

\textbf{Introduction for the final written report}

\begin{verbatim}
Cal State Northridge - Kinesiology
KIN-610 - Dr. Furtado
Points ___/20

YOUR RESEAERCH QUESTION
Type here...

Type your summary here...

Note: This assignment will become the "Introduction" section of your final 
project's written report.
\end{verbatim}

\hypertarget{written-report-1}{%
\subsection{Written report}\label{written-report-1}}

Similar to the Presentation Template, students are required to use a template created to write the report. The template is based on APA 6th ed.~and can be \href{project-template.docx}{downloaded here}.

\hypertarget{appendices}{%
\section{Appendices}\label{appendices}}

\hypertarget{appendix-a}{%
\subsection{A: Correlation criteria}\label{appendix-a}}

When evaluating the size of a bivariate correlation, please use Cohen (1988)

\begin{longtable}[]{@{}ll@{}}
\toprule()
Coefficient Value & Strength of Association \\
\midrule()
\endhead
0.1 \textless{} \emph{r} \textless{} .3 & small correlation \\
0.3 \textless{} \emph{r} \textless{} .5 & medium/moderate correlation \\
\emph{r} \textgreater{} .5 & Large/strong correlation \\
\bottomrule()
\end{longtable}

\hypertarget{appendix-b}{%
\subsection{B: Research Questions}\label{appendix-b}}

\hypertarget{rq-1}{%
\subsubsection{RQ \#1}\label{rq-1}}

It has been reported that players from \texttt{Iowa\ State} work harder during practice compared to players from any other college teams in the country.

Since you had access to players from another university in the State of Iowa \texttt{(University\ of\ Iowa)}, you decided to test this hypothesis.

As a group, do players from \texttt{Iowa\ State\ (70)} perform better than players from \texttt{Iowa\ University\ (69)} on the \textbf{Bench Press} (\texttt{bench\_press\_post)} test? How about for \textbf{vertical Leap} (\texttt{vert\_leap\_post})?

\begin{center}\rule{0.5\linewidth}{0.5pt}\end{center}

You will need to filter cases to perform this analysis so that only values \texttt{69} and \texttt{70} is selected for \texttt{college}. Notice that you will need to run the analysis for bench press and vertical leap.

\begin{center}\rule{0.5\linewidth}{0.5pt}\end{center}

\hypertarget{rq-2}{%
\subsubsection{RQ \#2}\label{rq-2}}

Research has shown that in the general population there is a \texttt{negative} and \texttt{moderate\ to\ high} correlation between \texttt{weight} and \texttt{performance\ on\ the\ test\ of}Broad Jump`. In other words, the heavier the person, the poorer the performance on the test, and vice-versa.

Assume that you are specially interested in football players that play as \texttt{Defensive\ Ends\ (3)}. Is there a relationship between \texttt{Weight} and \texttt{Broad\ Jump\ (pre)} scores among \texttt{Defensive\ Ends\ (3)}?

\begin{center}\rule{0.5\linewidth}{0.5pt}\end{center}

You will need to filter cases to perform this analysis so that only value \texttt{3} is selected/checked for \texttt{position}.

\begin{center}\rule{0.5\linewidth}{0.5pt}\end{center}

\hypertarget{rq-3}{%
\subsubsection{RQ \#3}\label{rq-3}}

Defensive players are known to be stronger than those playing on other positions. But how about if we compare defensive players among themselves from different positions?

Are Defensive \texttt{Tacklers\ (4)}, \texttt{Linebackers\ (10)}, and \texttt{Cornerbacks\ (2)} different when it comes to \texttt{bench\ press\ (pre)} scores?

\begin{center}\rule{0.5\linewidth}{0.5pt}\end{center}

You will need to filter cases to perform this analysis so that only the values \texttt{4}, \texttt{10} and \texttt{2} are selected/checked for \texttt{position}.

\begin{center}\rule{0.5\linewidth}{0.5pt}\end{center}

\hypertarget{rq-4}{%
\subsubsection{RQ \#4}\label{rq-4}}

As an athletic trainer working for the NFL Scouting Combine, you decided to test the effectiveness of a program you developed to improve players' \texttt{agility}. If deemed effective, the program could eventually be sold to NFL professional teams.

To test the effectiveness of the program, you invited players from \texttt{Ohio\ State\ University} to participate in this 2-week program (3 hours everyday). At the end of the 2-week period, players were re-tested on the \texttt{20-yard\ Shuttle} and the \texttt{3-cone\ Drill}.

Do players from \texttt{Ohio\ State\ University\ (127)} improve their scores on the \texttt{20-yard\ Shuttle} from pre to post-test? How about for the \texttt{3-cone\ Drill}?

\begin{center}\rule{0.5\linewidth}{0.5pt}\end{center}

You will need to filter cases to perform this analysis so that only the value \texttt{127} is selected/checked for \texttt{college}. Notice that you will need to run the analysis for 20-yard Shuttle and 3-cone Drill.

\begin{center}\rule{0.5\linewidth}{0.5pt}\end{center}

\hypertarget{rq-5}{%
\subsubsection{RQ \#5}\label{rq-5}}

There is evidence that \texttt{Wide\ Receivers} and \texttt{Running\ Backs} have a higher incidence of concussion compared to players from other football positions. Over the years, this may negatively affect the players' cognitive ability. For instance, how would they compare to \texttt{Quarterbacks}, who arguably are less likely to suffer concussions during a football match.

How do players playing as \texttt{Wide\ Receivers\ (22)}, \texttt{Running\ Backs\ (18)} and \texttt{Quarterbacks\ (17)} compare on the Wonderlic scores? For this analysis, please, \textbf{ONLY use the data for 2020 data}.

\begin{center}\rule{0.5\linewidth}{0.5pt}\end{center}

You will need to filter cases to perform this analysis so that only the values \texttt{17}, \texttt{18} and \texttt{22} are selected/checked for \texttt{position}.

\begin{center}\rule{0.5\linewidth}{0.5pt}\end{center}

\hypertarget{rq-6}{%
\subsubsection{RQ \#6}\label{rq-6}}

Quarterbacks must have excellent decision-making skills and act quickly under pressure during game plays.

Since the Wonderlic\footnote{\url{https://en.wikipedia.org/wiki/Wonderlic_test}} test assesses cognitive ability under pressure, an interesting question is whether QBs are above average when it comes to cognitive ability.

According to the test developers, the average (mean) score on the Wonderlic test is 20 and the median score is (19).

Do QBs tested in 2018, 2019, and 2019 perform better than the general population on the Wonderlic Cognitive ability test?

\begin{center}\rule{0.5\linewidth}{0.5pt}\end{center}

You will need to filter cases to perform this analysis so that only the value \texttt{17}, is selected/checked for \texttt{position}.

\begin{center}\rule{0.5\linewidth}{0.5pt}\end{center}

\hypertarget{rq-7}{%
\subsubsection{RQ \#7}\label{rq-7}}

It has been \href{https://www.cbssports.com/nfl/news/nfl-draft-combine-the-highest-and-lowest-wonderlic-test-scores-ever-recorded/}{reported} that as a group, \texttt{Offensive\ Tackles} perform better than \texttt{Full\ Backs} on the \texttt{Wonderlic\ test}.

Test the hypothesis that OT (15) players tested in 2018, 2019, and 2020 perform differently than FB (6) players on the Wonderlic Cognitive Ability test.

\begin{center}\rule{0.5\linewidth}{0.5pt}\end{center}

You will need to filter cases to perform this analysis so that only the values \texttt{5} and \texttt{15} are selected/checked for \texttt{position}.

\begin{center}\rule{0.5\linewidth}{0.5pt}\end{center}

\hypertarget{rq-8}{%
\subsubsection{RQ \#8}\label{rq-8}}

In the general population, there is a strong positive correlation between \texttt{weight} and \texttt{speed}. In other words, the heavier the player the slower the individual is, and vice-versa.

Considering that defensive players do train speed during football practices, it would be interesting to verify whether college players have a similar pattern compared to the general population.

Is there a correlation between \texttt{weight} and \texttt{speed} among football \texttt{Defensive\ Tackles\ (4)}? How about \texttt{Defensive\ Ends\ (3)}?

\begin{center}\rule{0.5\linewidth}{0.5pt}\end{center}

For this analysis, use the variables \texttt{Weight} and \texttt{40-yard\ Dash\ Pretest}, and filter cases so that only positions \texttt{3} and \texttt{4} are selected/checked.

\begin{center}\rule{0.5\linewidth}{0.5pt}\end{center}

\hypertarget{rq-9}{%
\subsubsection{RQ \#9}\label{rq-9}}

As a Motor Behaviorist, you work for the NFL Scouting Combine and see the opportunity to collect data and test some of the research questions you have in mind.

In 2020, you designed a program to help players improve their \texttt{speed}.

\texttt{Quarterbacks\ (17)} from all attending colleges were selected to be part of the intervention. Test the hypothesis that the players would improve from pre to post-test on the \texttt{40-yard\ Dash}.

\begin{center}\rule{0.5\linewidth}{0.5pt}\end{center}

For this analysis, use the variables \texttt{40-yard\ Dash}, and filter cases so that only position \texttt{17} is selected/checked. Also, note that this study involves only players who were tested in \texttt{2020}.

\begin{center}\rule{0.5\linewidth}{0.5pt}\end{center}

\hypertarget{rq-10}{%
\subsubsection{RQ \#10}\label{rq-10}}

As a Motor Behaviorist, you work for the NFL Scouting Combine and see the opportunity to collect data and test some of the research questions you have in mind.

In \texttt{2020}, you designed a program to help players improve their \texttt{speed} scores.

\texttt{Running\ Backs\ (18)} from all attending colleges were selected to be part of the intervention. Test the hypothesis that the players would improve from pre to post-test on the \texttt{40-yard\ Dash}.

\begin{center}\rule{0.5\linewidth}{0.5pt}\end{center}

For this analysis, use the variable \texttt{40-yard\ Dash}, and filter cases so that only position \texttt{18} is selected/checked. Also, note that this study involves only players who were tested in \texttt{2020}.

\begin{center}\rule{0.5\linewidth}{0.5pt}\end{center}

\hypertarget{rq-11}{%
\subsubsection{RQ \#11}\label{rq-11}}

As a Motor Behaviorist, you work for the NFL Scouting Combine and see the opportunity to collect data and test some of the research questions you have in mind.

In 2020, you designed a program to help players improve their \texttt{agility}.

\texttt{Quarterbacks\ (17)} from all attending colleges were selected to be part of the intervention. Test the hypothesis that the players would improve from pre to post-test on the \texttt{3-cone\ Drill}.

\begin{center}\rule{0.5\linewidth}{0.5pt}\end{center}

For this analysis, use the variable \texttt{3-cone\ Drill}, and filter cases so that only position \texttt{17} is selected/checked. Also, note that this study involves only players who were tested in \texttt{2020}.

\begin{center}\rule{0.5\linewidth}{0.5pt}\end{center}

\hypertarget{rq-12}{%
\subsubsection{RQ \#12}\label{rq-12}}

As a Motor Behaviorist, you work for the NFL Scouting Combine and see the opportunity to collect data and test some of the research questions you have in mind.

In 2020, you designed a program to help players improve their \texttt{agility}.

\texttt{Running\ Backs\ (18)} from all attending colleges were selected to be part of the intervention. Test the hypothesis that the players would improve from pre to post-test on the \texttt{3-cone\ Drill}.

\begin{center}\rule{0.5\linewidth}{0.5pt}\end{center}

For this analysis, use the variable \texttt{3-cone\ Drill}, and filter cases so that only position \texttt{18} is selected/checked. Also, note that this study involves only players who were tested in \texttt{2020}.

\begin{center}\rule{0.5\linewidth}{0.5pt}\end{center}

\hypertarget{rq-13}{%
\subsubsection{RQ \#13}\label{rq-13}}

In football, the size of a player's hand may make a difference at a decisive moment during a game. Test the hypothesis that \texttt{Wide\ Receivers\ (WR)}, \texttt{Quarterbacks\ (QB)}, and \texttt{Defensive\ Lines\ (DT)} will show differences in the size of their hands. Your prediction is that \texttt{WR} and \texttt{QB} players will have significantly larger hands compared to \texttt{DT} players and \texttt{WR} and \texttt{QB} players will show no differences in the size of their hands when compared.

\hypertarget{rq-14}{%
\subsubsection{RQ \#14}\label{rq-14}}

Certain football positions require players to perform extraordinary moves. \texttt{Wide\ Receivers} (WR) must show blazing speed and strong hand-eye coordination. But in some situations during a football game, \texttt{WR} must jump over players who try to tackle them.

Test the hypothesis that \texttt{WR} will perform differently when compared to \texttt{Quarterbacks} (QB) and \texttt{Defensive\ Line} (DT) players on the test of \texttt{Broad-Jump} (broad\_jump\_post). Your prediction is that \texttt{WR} will overperform both \texttt{QB} and \texttt{DL} players and the last two positions will not show difference when compared.

\begin{center}\rule{0.5\linewidth}{0.5pt}\end{center}

\emph{Hint: use \texttt{position\_labels} for this analysis.}

\begin{center}\rule{0.5\linewidth}{0.5pt}\end{center}

\hypertarget{deadlines}{%
\subsection{C: Deadlines}\label{deadlines}}

Refer to Canvas

\begin{center}\rule{0.5\linewidth}{0.5pt}\end{center}

\hypertarget{license}{%
\section{License}\label{license}}

KIN 610 Stats Course by Ovande Furtado Jr is licensed under CC BY-NC-SA 4.0

\end{document}
