% Options for packages loaded elsewhere
\PassOptionsToPackage{unicode}{hyperref}
\PassOptionsToPackage{hyphens}{url}
%
\documentclass[
]{article}
\usepackage{amsmath,amssymb}
\usepackage{lmodern}
\usepackage{iftex}
\ifPDFTeX
  \usepackage[T1]{fontenc}
  \usepackage[utf8]{inputenc}
  \usepackage{textcomp} % provide euro and other symbols
\else % if luatex or xetex
  \usepackage{unicode-math}
  \defaultfontfeatures{Scale=MatchLowercase}
  \defaultfontfeatures[\rmfamily]{Ligatures=TeX,Scale=1}
\fi
% Use upquote if available, for straight quotes in verbatim environments
\IfFileExists{upquote.sty}{\usepackage{upquote}}{}
\IfFileExists{microtype.sty}{% use microtype if available
  \usepackage[]{microtype}
  \UseMicrotypeSet[protrusion]{basicmath} % disable protrusion for tt fonts
}{}
\makeatletter
\@ifundefined{KOMAClassName}{% if non-KOMA class
  \IfFileExists{parskip.sty}{%
    \usepackage{parskip}
  }{% else
    \setlength{\parindent}{0pt}
    \setlength{\parskip}{6pt plus 2pt minus 1pt}}
}{% if KOMA class
  \KOMAoptions{parskip=half}}
\makeatother
\usepackage{xcolor}
\IfFileExists{xurl.sty}{\usepackage{xurl}}{} % add URL line breaks if available
\IfFileExists{bookmark.sty}{\usepackage{bookmark}}{\usepackage{hyperref}}
\hypersetup{
  pdftitle={Lab 3 - Comparing two means},
  pdfauthor={Furtado Jr., O.},
  hidelinks,
  pdfcreator={LaTeX via pandoc}}
\urlstyle{same} % disable monospaced font for URLs
\usepackage[margin=1in]{geometry}
\usepackage{longtable,booktabs,array}
\usepackage{calc} % for calculating minipage widths
% Correct order of tables after \paragraph or \subparagraph
\usepackage{etoolbox}
\makeatletter
\patchcmd\longtable{\par}{\if@noskipsec\mbox{}\fi\par}{}{}
\makeatother
% Allow footnotes in longtable head/foot
\IfFileExists{footnotehyper.sty}{\usepackage{footnotehyper}}{\usepackage{footnote}}
\makesavenoteenv{longtable}
\usepackage{graphicx}
\makeatletter
\def\maxwidth{\ifdim\Gin@nat@width>\linewidth\linewidth\else\Gin@nat@width\fi}
\def\maxheight{\ifdim\Gin@nat@height>\textheight\textheight\else\Gin@nat@height\fi}
\makeatother
% Scale images if necessary, so that they will not overflow the page
% margins by default, and it is still possible to overwrite the defaults
% using explicit options in \includegraphics[width, height, ...]{}
\setkeys{Gin}{width=\maxwidth,height=\maxheight,keepaspectratio}
% Set default figure placement to htbp
\makeatletter
\def\fps@figure{htbp}
\makeatother
\setlength{\emergencystretch}{3em} % prevent overfull lines
\providecommand{\tightlist}{%
  \setlength{\itemsep}{0pt}\setlength{\parskip}{0pt}}
\setcounter{secnumdepth}{-\maxdimen} % remove section numbering
\ifLuaTeX
  \usepackage{selnolig}  % disable illegal ligatures
\fi

\title{Lab 3 - Comparing two means}
\author{Furtado Jr., O.}
\date{10/5/2021}

\begin{document}
\maketitle

\textbf{Other Formats}: \href{ds1.pdf}{PDF} \textbar{}
\href{ds1.docx}{Word}

\hypertarget{learning-objectives}{%
\subsection{Learning objectives}\label{learning-objectives}}

\begin{enumerate}
\def\labelenumi{\arabic{enumi}.}
\tightlist
\item
  calculate measures of central tendency and variability in
  \texttt{jamovi}
\item
  differentiate between descriptive and inferential statistics
\item
  create and interpret histograms and boxplots
\item
  understand characteristics of normal and non-normal distributions
\end{enumerate}

\hypertarget{data-set}{%
\subsection{Data set}\label{data-set}}

We will use a modified version of the NFL Combine data set. You will
need to download the \texttt{csv} data file and open it with
\texttt{jamovi}.

\href{./datasets/data-labs.csv}{Click here} to download the data set for
this lab.

Note that you will be required to create filters\footnote{Creating
  filters in jamovi: \url{https://youtu.be/pij0KlFhITw}} when completing
some of the required analysis for this assignment.

\begin{longtable}[]{@{}ll@{}}
\caption{Position key}\tabularnewline
\toprule
Position ID & Name \\
\midrule
\endfirsthead
\toprule
Position ID & Name \\
\midrule
\endhead
1 & Cornerback \\
2 & Defensive lineman \\
3 & Safety \\
4 & Linebacker \\
5 & Offensive lineman \\
6 & Running back \\
7 & Wide receiver \\
\bottomrule
\end{longtable}

\hypertarget{problem-1}{%
\subsection{Problem 1}\label{problem-1}}

Suppose a researcher working for the NFL Combine Data Collection Team
wanted to find whether players of two different positions would differ
on their bench press scores (bpress). The field positions considered for
this study were: 1= Cornerback; 2= Defensive Lineman. The scores were
recorded as the number of repetitions @ 225 lbs. \textbf{Run an analysis
to learn if differences exist between these two positions on the bench
press variable.}

\hypertarget{question-1}{%
\subsubsection{Question 1}\label{question-1}}

\textbf{Filter}: create a filter for \texttt{position} so that only
\texttt{Wide\ Receivers} and \texttt{Safety} are included in this
analysis.

Calculate the \texttt{mean}, \texttt{standard\ deviation}, \texttt{n},
\texttt{skewness}, \texttt{kurtosis}, and \texttt{shapiro-wilk\ test}.
Ensure to provide the \texttt{Descriptives\ Table} as part of this
question.

\begin{verbatim}
 DESCRIPTIVES

 Descriptives                                            
 ─────────────────────────────────────────────────────── 
                         position             bpress     
 ─────────────────────────────────────────────────────── 
   N                     Cornerback                 15   
                         Defensive lineman          32   
   Mean                  Cornerback           16.86667   
                         Defensive lineman    28.15625   
   Standard deviation    Cornerback           3.044120   
                         Defensive lineman    5.941160   
 ─────────────────────────────────────────────────────── 
\end{verbatim}

\includegraphics{lab3-twomeans_files/figure-latex/unnamed-chunk-1-1.pdf}
\includegraphics{lab3-twomeans_files/figure-latex/unnamed-chunk-1-2.pdf}

Question 2

\hypertarget{question-2}{%
\subsection{Question 2}\label{question-2}}

\textbf{Filter}: turn off all filters before proceeding!

Considering the nature of both dependent variables used in
\texttt{Question\ 1} and assuming the distribution of scores for both
variables are \textbf{approximating normality}, which measure of central
tendency should be reported (mean, mode, or median)? Explain.

\hypertarget{question-3}{%
\subsection{Question 3}\label{question-3}}

In \texttt{question\ 1}, you calculated the measures of central tendency
and variability, which fall under the category of descriptive
statistics. Discuss the difference between \textbf{descriptive
statistics} and \textbf{inferential statistics}. More complete answers
will receive more points.

\hypertarget{question-4}{%
\subsection{Question 4}\label{question-4}}

\textbf{Filter}: create a filter to \texttt{Position} so that only
\texttt{Quarterbacks} (QB) are included in this analysis.

Using the variable \texttt{BroadJumpin}, calculate the \texttt{range},
\texttt{standard\ deviation} and the \texttt{IQR} for Quarterbacks (QB)
ONLY.

\hypertarget{question-5}{%
\subsection{Question 5}\label{question-5}}

\textbf{Filter}: turn off all filters before proceeding!

In \texttt{Question\ 1}, you were asked to calculate the standard
deviation of height (inches) and weight (lbs). Suppose you want to
compare the \texttt{standard\ deviations} of \texttt{height} and
\texttt{weight}. How the two standard deviations compare?

\hypertarget{question-6}{%
\subsection{Question 6}\label{question-6}}

When calculating the sample variance and standard deviation,
\texttt{jamovi} uses \texttt{N-1} in the denominator (see below). In
your own words, explain why \texttt{N-1} is used instead of \texttt{N}.

Standard Deviation equation for the sample:

\(\sigma = \sqrt{\frac{\sum (x - \mu)^2}{N-1}}\)

Variance equation for the sample:

\(s^2 = \frac{\sum (x - \bar{x})^2}{N - 1}\)

\hypertarget{question-7}{%
\subsection{Question 7}\label{question-7}}

The \texttt{variance} and the \texttt{standard\ deviation} are the two
most common measures of variability reported in research manuscripts.
Let's say the manuscript you submitted for publication was returned by
the editor. The editor-in-chief has asked you to report either the
\texttt{variance} OR the \texttt{standard\ deviation}. Which one would
you pick and why?

\hypertarget{question-8}{%
\subsection{Question 8}\label{question-8}}

\textbf{Filter}: create a filter to \texttt{Status} so that
\textbf{only} \texttt{Year\ 2} is included in this analysis.

The variable ``Status'' refers to players who were either tested during
the first or second year. Run an analysis to calculate the
\texttt{mean}, and \texttt{standard\ deviation} for the variable
\texttt{Shuttle}.

\hypertarget{question-9}{%
\subsection{Question 9}\label{question-9}}

Create a \texttt{histogram} for the variable \texttt{Shuttle} and add
the density line to it.

Provide the histogram create by \texttt{jamovi} below and state whether
the distribution of scores for \texttt{Shuttle} appears to be deviating
or approximating normality. In this particular case, disregard other
sources of normality (skewness, kurtosis, QQ-plots, Shapiro-Wilk test,
etc.).

\hypertarget{question-10}{%
\subsection{Question 10}\label{question-10}}

Create a \texttt{boxplot} for \texttt{Shuttle}. Did the creation of the
boxplot reveal any outliers? Explain.

\hypertarget{license}{%
\section{License}\label{license}}

This work is licensed under a Creative Commons
Attribution-NonCommercial-NoDerivatives 4.0 International License.

\end{document}
